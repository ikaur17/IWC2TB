%% Copernicus Publications Manuscript Preparation Template for LaTeX Submissions
%% ---------------------------------
%% This template should be used for copernicus.cls
%% The class file and some style files are bundled in the Copernicus Latex Package, which can be downloaded from the different journal webpages.
%% For further assistance please contact Copernicus Publications at: production@copernicus.org
%% https://publications.copernicus.org/for_authors/manuscript_preparation.html


%% Please use the following documentclass and journal abbreviations for preprints and final revised papers.

%% 2-column papers and preprints
\documentclass[amt, manuscript]{copernicus}



%% \usepackage commands included in the copernicus.cls:
%\usepackage[german, english]{babel}
%\usepackage{tabularx}
%\usepackage{cancel}
%\usepackage{multirow}
%\usepackage{supertabular}
%\usepackage{algorithmic}
%\usepackage{algorithm}
%\usepackage{amsthm}
%\usepackage{float}
%\usepackage{subfig}
%\usepackage{rotating}


\begin{document}

\title{Implications of ice hydrometeor orientation on IWP retrievals using GMI polarimetric measurements}


% \Author[affil]{given_name}{surname}

\Author[1]{Inderpreet}{Kaur}
\Author[1]{Patrick}{Eriksson}
\Author[1]{Vasileios}{Barlakas}
\Author[1]{Simon}{Pfreundschuh}


\affil[1]{Chalmers University of Technology, Gothenburg, Sweden}

%% The [] brackets identify the author with the corresponding affiliation. 1, 2, 3, etc. should be inserted.

%% If an author is deceased, please mark the respective author name(s) with a dagger, e.g. "\Author[2,$\dag$]{Anton}{Smith}", and add a further "\affil[$\dag$]{deceased, 1 July 2019}".

%% If authors contributed equally, please mark the respective author names with an asterisk, e.g. "\Author[2,*]{Anton}{Smith}" and "\Author[3,*]{Bradley}{Miller}" and add a further affiliation: "\affil[*]{These authors contributed equally to this work.}".


\correspondence{Inderpreet Kaur (kauri@chalmers.se)}

\runningtitle{Ice water path retrievals}

\runningauthor{Kaur}





\received{}
\pubdiscuss{} %% only important for two-stage journals
\revised{}
\accepted{}
\published{}

%% These dates will be inserted by Copernicus Publications during the typesetting process.


\firstpage{1}

\maketitle



\begin{abstract}
Existing datasets of ice water path (IWP) based on passive observations 
either show a clear low bias compared to radar retrievals or are 
completely empirical. It is here shown that much more accurate 
physically-based passive retrievals are possible. The high-frequency 
channels of the GMI radiometer are used for demonstration. The progress 
is mainly achieved by detailed radiative transfer simulations, that 
closely mimic the real observations statistically. A novel aspect is 
that particle orientation is considered and polarisation signatures in 
the observations can be exploited. The simulations are used as input to 
a machine learning retrieval algorithm. Both the simulations and the 
inversions are made in a Bayesian fashion. As a consequence, reasonably 
complete distributions of IWP are provided and the retrievals can be 
considered as all-sky in contrast to most other datasets, limited to 
either the low or high end of the IWP distribution. Example results and 
applications are shown. For example, a clear diurnal variation of IWP in 
the Tropics is confirmed. So far satellite-based cloud radars are only 
nadir-pointing while these passive retrievals have a swath width of ? 
km, and this broad coverage should provide opportunities to assess model 
cloud parametrisations in new manners

\end{abstract}


\copyrightstatement{TEXT} %% This section is optional and can be used for copyright transfers.


\introduction  %% \introduction

Ice Clouds have a profound impact the Earth's radiation balance. They reflect the short and long wave radiation and emit long wave radiation contributing to the Earth’s energy balance and hydrological cycle \citep{liou:influ:86}. The total radiative forcing is dependent both on the microphysical and macrophysical properties, such as, the spatial distribution of ice particles, particle orientation, ice water content, etc. However, due to the complex vertical and spatio-temporal heterogeneity associated with ice clouds, the knowledge on the global distribution of atmospheric ice is deficient and not well captured by the models \citep{wilson:theim:00}. Significant uncertainties, in both meteorological and climate models, are inflicted by the simplifications introduced to resolve the complex structures \citep{reinhardt:impac:04}.

Ice water content (IWC)  is a key variable used to measure the atmospheric ice. It is defined as the bulk mass of ice in the atmosphere and the column integrated bulk mass is the ice water path (IWP). The existing space based retrieval systems use observations from either active or passive microwave sensors or exploit the synergies between the two. The 94\,\,GHz radar onboard Cloudsat is the first active sensor to give information on the vertical structure of clouds on a global scale. The synergy of the radar and CALIPSO lidar is the most accurate observation system providing global distribution of atmospheric ice, especially for the tropical ice clouds \citep{protat:theev:10}. This combination can sense ice hydrometeors ranging from thin cirrus to precipitating ice \citep{stephens:cloud:18}. However, the limited spatio-temporal sampling cannot fully resolve the  variability of atmospheric ice on both local and global scales.

Despite the lack of scanning capability and single-frequency sensor, the researchers have benefited from tremendously from Cloudsat. Multiple IWP and IWC retreival algorithms based on passive sensors but constrained by Cloudsat observations have evolved. For example, \citet{gong:cloud:14} describe an empirical model to retreive IWP from Microwave Humidty Sounder (MHS) high frequency channels. The Synergistic Passive Atmospheric Retrieval Experiment-ICE (Spare-Ice) product \citet{holl:spare:14} is another empirical attempt which combines optical,microwave and infrared sensors to retrieve IWP. The usage infrared sensors can account for smaller ice particles in the cloud, they cannot sense the entire water column. 

The global atmopsheric ice estimates from passive microwave sensors have proven to be successful capturing the diurnal cycle and interseasonal fluctuations. However, the inconsistency among different cloud ice observations leads to model uncertainties. A detailed comparison of different space borne IWP measurements by \citet{duncan:anupd:18, eliasson:asses:11} highlights that large discrepancies exist between different IWP estimates as the space-borne remote sensing techniques still cannot fully resolve the atmospheric ice properties. The limitations associated with sensor sensitivities and the oversimplified microphysical assumptions are mostly to blame. The sensor limitations will be overcome to a certain extent with the launch of new sensors like Ice Cloud Imager (ICI). ICI will will sense ice-clouds using sub-millimter(sub-mm) wavelengths and shall shall bridge the sensitivity gap between microwave and infrared sensors. However, the improvements in microphysical assumptions has seen slow progress. The physically based retrieval techniques associate the simulated brightness temperatures to the ice cloud structures, but most radiative transfer based retrieval algorithms, focusing on ice clouds, assume spherical or randomly oriented hydrometeors, e.g. \citep{evans:icecl:05, Zhao:retri:02}. The simplifications are often introduced to reduce computational loads, but at the same time, properties of ice-hydrometeors are still not fully understood. The ice hydrometeors exhibit dichroism effects resulting in polarization difference (PD) between the radiances measured by horizontal(H) and vertical(V) polarizations. The PD exhibited by oriented particles is generally higher than the randomly oriented ones. Constraining the microphysical assumptions to one or few habits and neglecting  orientation effects contributes to a non-zero bias in the retrievals. High inaccuracies also are connected to areas with large PD. For example, \citet{gong:micro:17} and \citep{miao:thepo:03} argue that retrieval of cloud ice has a strong connection to the polarimetric difference introduced by oriented particles, particularly for regions with convective outflow. 

Zeroing on an ice model incorporating accurate ice model remains a challenge, however, microwave sensors measuring different polarizations can help in inferring the properties of ice hydrometeors. The space borne sensors can provide accurate measures of the polarimetric differences and their potential in retrieval of ice cloud properties have have been long recognised. For example, \citep{coy:sensi:20}, have shown  that incorporating polarimetric measurements improve the retrievals of ice cloud microphysics such as particle effective diameter (D$_{eff}$), while \citep{hioki:degre:16} analyse the ice particle surface roughness from polarimetric observations. However, there is a general lack of  IWC and IWP retrieval methods taking PD into account.

In this study, we explore the benefits of incorporating polarimetric measurements for retrievals of IWP using Global Precipitation Measurement (GPM) Microwave Imager (GMI). We aim for physical based measurements of IWP from passive microwave instruments, which can be bias-free and better performing than existing reanalyses and satellite based products. Currently, GMI is the only sensor which senses in dual polarisation mode above 100\,\,GHz. The Bayesian retrieval algorithm utilized by GPM for precipiation is also used for compute IWP from precipitating hydrometeors. The ice hydrometeors are not covered as the apriori database is generated for hydrometeor profiles covered by GPM Dual Precipitraion radar (DPR), which is sensitive only to precipitating hydrometeors. 

We apply a Bayesian machine learning algorithm, Quantile Regression Neural Network (QRNN) \citep{pfreundschuh:aneur:18} to retrieve IWP from a database of atmospheric profiles generated using radiative transfer model. As a first step, a database representing the GMI measurements for a variety of atmospheric conditions is created. We avoid the ice-scattering calculations but the impact of polarisation is introduced using the scheme proposed by \citet{barlakas:intro:21}. They had introduced a simple scaling factor to mimic the effects of hydrometeor orientation. In this study, the simple scaling factor is extended to cover different magnitudes of polarimetric signatures observed by the Global Precipitation Measurement (GPM) Microwave Imager (GMI) 166 GHz channels. Further, the retrieval is applied to real GMI measurements and the results are compared against the existing IWP observation sets. We investigate the sensitivity of retrieved IWP to polarization differences and  quantify the errors arsing due to assumption of unpolarised signals. 

\section{Satellite and model data}

\subsection{GPM GMI}
The GPM GMI instrument is a conical microwave radiometer, which provides a near global coverage of the precipitation estimates. It has a swath of 885\,\,km  and an earth incidence angle of 52.8$^{\circ}$. It instrument has thirteen microwave channels ranging from 10\,\,GHz to 183\,\,GHz which are sensitive to the different forms of precipitation. In this study, we use only the four high frequency measurements between 166 GHz and 183 GHz (Table~\ref{tab:gmi_channels}). These channels are sensitive to the precipitating ice and water vapour.
\subsubsection{L1B radiances}

In this study, the retrieval is made from L1B radiances. The details of L1B algorithm are found in \citet{}. L1B radiances are gelocated and calibrated Level 0 counts at native resolution. 

\subsubsection{L3 IWP product}
For comparison, we also utilize GMI IWP product.  


%\subsection{Cloudsat}
%CloudSat is a sun-synchronous satellite which is part of the A-train constellation and has been providing information on the vertical structure of clouds and aerosols, since 2006. It carries a 96\,\,GHz Cloud Profiling Radar (CPR) to measure the backscattering by clouds. It has a vertical resolution of 485\,\,km, and 1.8\,\,km along-track and 1.2\,\,km cross-track resolution. 

%\subsection{ERA5}

%ERA5 is global reanalysis data provided by European Centre for Medium-Range Weather Forecasts (ECMWF). The reanalysis combines model data with observations to provide hourly estimates of a large number of atmospheric, ocean and land-surface variables . In this study, we use multiple ERA5 atmospheric and surface parameters for generation of atmospheric scenes to be used as inputs to radiative transfer simulations described in sect.~\ref{sec:atm_scenes}. The parameters used here are temperature, cloud liquid water content, specific humidity, 2\,\,m temperature, 10\,\,m U wind, 10\,\,V wind, land sea mask, snow depth, sea ice cover, surface elevation. 


\subsection{Atmospheric scenarios}
\label{sec:atm_scenes}

A bulk database of two-dimensional atmospheric scenes is created based on Cloudsat radar reflectivities \citep{marchand:hydro:08} and ERA5 atmospheric and surface parameters \citep{era5:18}. The radar reflectivities are inverted to obtain vertical profiles of ice hydrometeors and rain, by assuming a particle model(described in sect.~\ref{sec:arts_setup}), while the ERA5 data (e.g. temperature, cloud liquid water content, 10\,\,m wind, specific humidity, 2\,\,m temperature, ...) provide the background data.

ERA5 data are collocated in space and time with the ascending and descending orbits of Cloudsat. The latitudinal grid is derived from Cloudsat orbit, while a custom vertical grid from -700\,\,m to 20000\,\,m is defined. The vertical grid spacing is 125\,\,m upto 8\,\,km and 250\,\, beyond 8\,\,km. Land and water are classified using ERA5 land sea mask. Furthermore, the locations with snow depth larger than 0.002\,\,m are classified as snow, while seaice locations are randomly selected from the regions with non-zero sea ice cover. The randomness helps to avoid too coherent sea-ice regions among different synthetic scenes.
All values liquid water content below $10^{-8}$ or at temperature lower than 248\,\,K are also set to zero. Oxygen and Nitrogen levels are assumed to be varying with the amount of water vapour in the air. 

In 2011, CPR suffered from a battery anomaly and has operated in daylight-only mode since then. In order to avoid any bias due to limited temporal coverage, we select random CloudSat profiles from 2009. The data is constrained between $65^{\circ}$S and $65^{\circ}$N to match the GMI latitudinal coverage. 


\section{Tools and Methodology}

\subsection{Radiative Transfer Simulations}
\label{sec:rt_simulations}

\subsubsection{ARTS setup}
\label{sec:arts_setup}
All forward simulations are performed with Atmospheric Radiative Transfer System \citep{eriksson:arts2:11}. The two-dimensional atmosphere is approximated using an independent beam approximation as utilized in works by \citet{ekelund2020using, eriksson:towar:20}. 
The radar reflectivities are converted to bulk properties: IWC and rain water content (RWC), using a dbZ-based model system as described by \citet{ekelund2020using}. The microphysical assumptions are described below in Sect.~\ref{sec:hydrometeor_prop}. The temperature field is used to make the differentiation between liquid and ice hydrometeors. All hydrometeors below 0$^{\circ}$ are classified as ice and vice-versa. The radar measurements are inverted using an onion peeling concept. The method  divides the atmosphere into layers and for each layer the measurements are inverted using an IWC-dbZ table. The attenuation effects from hydrometeors and absorbing specied are also considered. To avoid spurious water contents due to surface clutter, a clutter zone at 1.2\,\,km land and 0.6\,\,km over water is defined. All retrievals below this zone are rejected and instead the value just above the clutter zone is used as a fill value. Additionally, all retrievals above .... are assumed to be erroneous and rejected.   

The land and water/ocean emissivities are taken from Tool to Estimate Land-Surface Emissivities at Microwave frequencies (TELSEM; \citep{aires2011tool} and  Tool to Estimate Sea-Surface Emissivity from Microwaves to sub-Millimeter waves (TESSEM;
\citep{prigent2017sea}, respectively. The snow and seaice emissivities are calculated using an empirical emissivity model described in Sect.~\ref{sec:snow_emissivity}.

All simulations are performed in ``all-sky'' mode with 53$^{\circ}$ incidence angle. Only high frequency GMI channels, ie. 166V, 166H, 183$\pm$3, 183$\pm$7 are simulated. The GMI antenna pattern is assumed to be a Gaussian footprint with full width at half maximum of 7\,\,km. Multiple pencil beam simulations are averaged over this pattern to yield the antenna brightness temperatures.   

\subsubsection{Hydrometeor properties}
\label{sec:hydrometeor_prop}

Three types of hydrometeors are considered for the radiative transfer calculation: cloud ice, rain and liquid clouds. For inversion of Cloudsat reflectivities to IWC, two different ice habits are applied: large plate aggregate and evans snow aggregate. Both particle habits are applied separately to generate two databases.
The single scattering data are taken from ARTS single scattering database provided by \citet{eriksson:agene:18} and the particle size distribution (PSD) developed by \citet{field:snows:07} is used. F07 has two different distributions for tropics and midlatitudes, but here we have used the tropical setting, further denoted as F07T. 

%Since the accuracy of the Cloudsat inversions are largely dependent on the assumed microphysical properties, a comparison of the databases generated with the two ice habits is made.

Rain is assumed as a liquid sphere and the PSD is calculated using Marshall Palmer size distribution \citep{marshall:thedi:48}. The liquid clouds is assumed to be totally absorbing and their scattering effects neglected. For liquid clouds the absorption model by \citet{ellison2007permittivity} is followed.  


\subsubsection{Polarisation correction}
\label{sec:scaling_factor}

In this study, both polarised and randomly oriented hydrometeors are simulated.  Including oriented particles inside radiative transfer is computationally expensive, however  \citet{barlakas:intro:21} have shown that the observed polarisation effects introduced by the oriented particles can be mimicked by a simple scaling factor. The layer optical thickness in H- and V- polarised channel is increased and decreased respectively by a correction factor $\alpha$, and the ratio of the corrected optical thickness between H- and V- channels is polarisation ratio ($\rho$) :
\begin{eqnarray}
\rho = \frac{1+\alpha}{1-\alpha}
\end{eqnarray}

Here, a similar approach is followed with slight modifications. Firstly, we scale the hydrometeor optical properties for H- and V- polarisations. And secondly, instead of applying one scaling factor to the entire data, we make a random selection for $rho\in[1, 1.4]$. A slightly smaller scaling is also applied to radar back-scattering. 


\subsubsection{Snow emissivity model}
\label{sec:snow_emissivity}

The emissivities over land and water are taken from climatologies, as mentioned above. However, for snow and sea ice surface types an empirical snow
and ice emissivity model was developed, based on the studies of
\citet{harlow:2009:milli, harlow:2012:tundr} and \citet{hewison:2002:airbo}. If
$\epsilon_{193}, \epsilon_{159}$ represents the emissivities for 193\,GHz and
159\,GHz respectively, then
\begin{align}
\epsilon_{193}& = \min({N(\mu_{193}, \sigma_{193}^{2}), 1});\, \mu_{193} = 0.78, \sigma_{193} = 0.07 \label{eq:1}\\
\epsilon_{159}& = \min(\epsilon_{193} - N(\mu_{159}, \sigma_{159}^{2}), 1) ;\,  \mu_{159} = 0.02, \sigma_{159} = 0.02\,\label{eq:2}
\end{align}
where, $N(\mu, \sigma^{2})$ represents the standard normal distribution with
mean $\mu$ and standard deviation $\sigma$. The differences between the
horizontal and vertical polarisations for both frequencies are 
approximated through a uniform random distribution.

\begin{align}
d_{159}& = U(a_1, b_1) ;\, a_1 = 0.005, b_1 = 0.055\\
d_{193}& = d_{159} - U(a_2, b_2) ;\, a_2 = 0.015, b_2 = 0.025 \,
\end{align}
where, $U(a, b)$ represents a uniform distribution between a and b. 


\subsection{Retrieval Algorithm}
\label{sec:retrieval_algo}






\subsection{Evaluation}

\section{Results}
\subsection{Simulations}

\subsection{Retrieval results: Test data}

\subsection{Retrieval results: GMI observations}

\section{Discussion}


\conclusions  %% \conclusions[modified heading if necessary]
TEXT

%% The following commands are for the statements about the availability of data sets and/or software code corresponding to the manuscript.
%% It is strongly recommended to make use of these sections in case data sets and/or software code have been part of your research the article is based on.

\codeavailability{TEXT} %% use this section when having only software code available


\dataavailability{TEXT} %% use this section when having only data sets available


\codedataavailability{TEXT} %% use this section when having data sets and software code available


\sampleavailability{TEXT} %% use this section when having geoscientific samples available


\videosupplement{TEXT} %% use this section when having video supplements available


\appendix
\section{}    %% Appendix A

\subsection{}     %% Appendix A1, A2, etc.


\noappendix       %% use this to mark the end of the appendix section. Otherwise the figures might be numbered incorrectly (e.g. 10 instead of 1).

%% Regarding figures and tables in appendices, the following two options are possible depending on your general handling of figures and tables in the manuscript environment:

%% Option 1: If you sorted all figures and tables into the sections of the text, please also sort the appendix figures and appendix tables into the respective appendix sections.
%% They will be correctly named automatically.

%% Option 2: If you put all figures after the reference list, please insert appendix tables and figures after the normal tables and figures.
%% To rename them correctly to A1, A2, etc., please add the following commands in front of them:

\appendixfigures  %% needs to be added in front of appendix figures

\appendixtables   %% needs to be added in front of appendix tables

%% Please add \clearpage between each table and/or figure. Further guidelines on figures and tables can be found below.



\authorcontribution{TEXT} %% this section is mandatory

\competinginterests{TEXT} %% this section is mandatory even if you declare that no competing interests are present

\disclaimer{TEXT} %% optional section

\begin{acknowledgements}
TEXT
\end{acknowledgements}




%% REFERENCES

\bibliographystyle{copernicus}
\bibliography{references.bib}

\begin{thebibliography}{}

\bibitem[AUTHOR(YEAR)]{LABEL1}
REFERENCE 1

\bibitem[AUTHOR(YEAR)]{LABEL2}
REFERENCE 2

\end{thebibliography}

%% Since the Copernicus LaTeX package includes the BibTeX style file copernicus.bst,
%% authors experienced with BibTeX only have to include the following two lines:
%%
%% \bibliographystyle{copernicus}
%% \bibliography{example.bib}
%%
%% URLs and DOIs can be entered in your BibTeX file as:
%%
%% URL = {http://www.xyz.org/~jones/idx_g.htm}
%% DOI = {10.5194/xyz}


%% LITERATURE CITATIONS
%%
%% command                        & example result
%% \citet{jones90}|               & Jones et al. (1990)
%% \citep{jones90}|               & (Jones et al., 1990)
%% \citep{jones90,jones93}|       & (Jones et al., 1990, 1993)
%% \citep[p.~32]{jones90}|        & (Jones et al., 1990, p.~32)
%% \citep[e.g.,][]{jones90}|      & (e.g., Jones et al., 1990)
%% \citep[e.g.,][p.~32]{jones90}| & (e.g., Jones et al., 1990, p.~32)
%% \citeauthor{jones90}|          & Jones et al.
%% \citeyear{jones90}|            & 1990



%% FIGURES

%% When figures and tables are placed at the end of the MS (article in one-column style), please add \clearpage
%% between bibliography and first table and/or figure as well as between each table and/or figure.

% The figure files should be labelled correctly with Arabic numerals (e.g. fig01.jpg, fig02.png).


%% ONE-COLUMN FIGURES

%%f
%\begin{figure}[t]
%\includegraphics[width=8.3cm]{FILE NAME}
%\caption{TEXT}
%\end{figure}
%
%%% TWO-COLUMN FIGURES
%
%%f
%\begin{figure*}[t]
%\includegraphics[width=12cm]{FILE NAME}
%\caption{TEXT}
%\end{figure*}
%
%
%%% TABLES
%%%
%%% The different columns must be seperated with a & command and should
%%% end with \\ to identify the column brake.
%
%%% ONE-COLUMN TABLE
%
%%t
%\begin{table}[t]
%\caption{TEXT}
%\begin{tabular}{column = lcr}
%\tophline
%
%\middlehline
%
%\bottomhline
%\end{tabular}
%\belowtable{} % Table Footnotes
%\end{table}
%
%%% TWO-COLUMN TABLE
%
%%t
%\begin{table*}[t]
%\caption{TEXT}
%\begin{tabular}{column = lcr}
%\tophline
%
%\middlehline
%
%\bottomhline
%\end{tabular}
%\belowtable{} % Table Footnotes
%\end{table*}
%
%%% LANDSCAPE TABLE
%
%%t
%\begin{sidewaystable*}[t]
%\caption{TEXT}
%\begin{tabular}{column = lcr}
%\tophline
%
%\middlehline
%
%\bottomhline
%\end{tabular}
%\belowtable{} % Table Footnotes
%\end{sidewaystable*}
%
%
%%% MATHEMATICAL EXPRESSIONS
%
%%% All papers typeset by Copernicus Publications follow the math typesetting regulations
%%% given by the IUPAC Green Book (IUPAC: Quantities, Units and Symbols in Physical Chemistry,
%%% 2nd Edn., Blackwell Science, available at: http://old.iupac.org/publications/books/gbook/green_book_2ed.pdf, 1993).
%%%
%%% Physical quantities/variables are typeset in italic font (t for time, T for Temperature)
%%% Indices which are not defined are typeset in italic font (x, y, z, a, b, c)
%%% Items/objects which are defined are typeset in roman font (Car A, Car B)
%%% Descriptions/specifications which are defined by itself are typeset in roman font (abs, rel, ref, tot, net, ice)
%%% Abbreviations from 2 letters are typeset in roman font (RH, LAI)
%%% Vectors are identified in bold italic font using \vec{x}
%%% Matrices are identified in bold roman font
%%% Multiplication signs are typeset using the LaTeX commands \times (for vector products, grids, and exponential notations) or \cdot
%%% The character * should not be applied as mutliplication sign
%
%
%%% EQUATIONS
%
%%% Single-row equation
%
%\begin{equation}
%
%\end{equation}
%
%%% Multiline equation
%
%\begin{align}
%& 3 + 5 = 8\\
%& 3 + 5 = 8\\
%& 3 + 5 = 8
%\end{align}
%
%
%%% MATRICES
%
%\begin{matrix}
%x & y & z\\
%x & y & z\\
%x & y & z\\
%\end{matrix}
%
%
%%% ALGORITHM
%
%\begin{algorithm}
%\caption{...}
%\label{a1}
%\begin{algorithmic}
%...
%\end{algorithmic}
%\end{algorithm}
%
%
%%% CHEMICAL FORMULAS AND REACTIONS
%
%%% For formulas embedded in the text, please use \chem{}
%
%%% The reaction environment creates labels including the letter R, i.e. (R1), (R2), etc.
%
%\begin{reaction}
%%% \rightarrow should be used for normal (one-way) chemical reactions
%%% \rightleftharpoons should be used for equilibria
%%% \leftrightarrow should be used for resonance structures
%\end{reaction}
%
%
%%% PHYSICAL UNITS
%%%
%%% Please use \unit{} and apply the exponential notation


\end{document}
